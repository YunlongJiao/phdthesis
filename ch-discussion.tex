\chapter{Discussion}
\label{chap:discussion}

% th�se, d�finir

Thesis contributions:

- Theory

- Methodology

- Software

Research perspectives:

- machine learning

* theoretical guarantees on rankings
* constitutes an elegant confirmation of the concept of using ranking-related methods for bioinformatics but more ideas to motivate by social choice theory etc
* other cross-subject directions of future research such as pure math (geometric group theory, algebraic group theory, etc) and other biomedical application

- bioinformatics

* far from clinical
* Although a few gene expression-based breast cancer prognostic predictors have proved successful in clinical implementation, the fact that a large number of identified signatures failed to surpass the performance of conventional clinical covariates raises the controversial topic suspecting the incorporation of molecular data in application can be impractical. Sage-DREAM BCC results...
* subtyping of breast cancers and prognosis/treatment for each subtype
* wet-lab experimental validation

% ch2-3 embedding for permutations ?!
% other embedding - just (1,\dots,n) or permutation matrix - high-order embedding
% what is the convex hull of Kendall embedding for
%Check Fogel (NIPS, 2013) and Stephen Wright (NIPS, 2014) for two embeddings of symmetric group $\mathbb{S}_n$ into Euclidean spaces, former into dim $n^2$ while latter into dim $n-1$ (or a hyperplane in dim $n$), thus facilitating the convex relaxation to seriation problems by solving an optimization problem constrained in a convex hull, former in Birkhoff polytope
%$$\mathcal{B}:=\{\mathbf{X}\in \mathbb{R}^{n\times n}:\mathbf{X}\succeq 0,\mathbf{X}\mathbf{1}=\mathbf{1},\mathbf{X}^T \mathbf{1}=\mathbf{1}\}$$
%while latter in permutahedron
%$$\mathcal{PH}:=\{x\in \mathbb{R}^n:\sum_{i=1}^n x_i=\frac{n(n+1)}{2},\sum_{i\in S}x_i \leq \sum_{i=1}^{|S|}(n+1-i)\mbox{ for all }S\subset [n]\}.$$
%Kendall kernel comes with another embedding into Euclidean space of dim $O(n^2)$. Several questions are
%\begin{enumerate}
%\item What is the convex hull induced by our embedding? What is the projection of any vector onto the convex hull (to facilitate optimization algorithm using projections)?
%\item On the other way round, what are the explicit embedding, kernel function and distance induced by those two well-known embeddings?
%\item Is it a better embedding compared to those two embeddings to introduce a convex hull for solving seriation problem, when we convert it into an optimization problem by following the same idea as Fogel (NIPS, 2013) after the embedding?
%\end{enumerate}

