\chapter{Signaling Pathway Activities Improve Prognosis for Breast Cancer}
\label{chap:hipathia}
%\minitoc

\begin{chapabstract}

\textrm{{\bf Abstract:}} With the advent of high-throughput technologies for genome-wide expression profiling, a large number of methods have been proposed to discover gene-based signatures as biomarkers to guide cancer prognosis. However, it is often difficult to interpret the list of genes in a prognostic signature regarding the underlying biological processes responsible for disease progression or therapeutic response. A particularly interesting alternative to gene-based biomarkers is mechanistic biomarkers, derived from signaling pathway activities, which are known to play a key role in cancer progression and thus provide more informative insights into cellular functions involved in cancer mechanism. In this chapter, we demonstrate that a pathway-level feature, such as the activity of signaling circuits, outperform conventional gene-level features in prediction performance in breast cancer prognosis. We also show that the proposed classification scheme can even suggest, in addition to relevant signaling circuits related to disease outcome, a list of genes that do not code for signaling proteins whose contribution to cancer prognosis potentially supplements the mechanisms detected by pathway analysis. This study is under submission as joint work with Marta R. Hidalgo, Cankut \c{C}ubuk, Alicia Amadoz, Jos\'{e} Carbonell-Caballero, Jean-Philippe Vert, Joaqu\'{i}n Dopazo \cite{Jiao2017Signaling}.
\linebreak
\vskip 0.1in
\noindent \textrm{{\bf R�sum�:}}

\end{chapabstract}



\section{Introduction}
\label{sec5:intro}

Over the past decades, many efforts have been addressed to the identification of gene-based signatures to predict  patient prognosis using gene expression data \cite{VantVeer2002Gene, Paik2004multigene, Wang2005Gene, Sotiriou2009Gene, Reis-Filho2011Gene}. Despite the success of its use, gene expression signatures have not been exempt of problems \cite{Ein-Dor2006Thousands, Iwamoto2010Predicting}. Specifically, one major drawback of multi-gene biomarkers is that they often lack proper interpretation in terms of mechanistic link to the fundamental cell processes responsible for disease progression or therapeutic response \cite{VantVeer2008Enabling, Dopazo2010Functional}. Actually, it is increasingly recognized that complex traits, such as disease or drug response, are better understood as alterations in the operation of functional modules caused by different combinations of gene perturbations \cite{Barabasi2004Network, Oti2007modular, Barabasi2011Network}. To address this inherent complexity different methodologies have tried to exploit several functional module conceptual representations, such as protein interaction networks or pathways, to interpret gene expression data within a systems biology context \cite{Barabasi2011Network, Vidal2011Interactome, Hood2013Systems, Fryburg2014Systems}.

Here we focus on consulting prior knowledge of signaling pathways to guide cancer prognosis. It is well understood that cell signaling is a system of within-cell communication and signal transduction process between gene products, mostly proteins, that coordinates cell activities to perceive and correctly respond to microenvironment, resulting in signaling pathways that form a particular type of functional gene modules and play a key role in disease progression (Figure \ref{fig5:cellsignaling}). Consequently as a tempting solution to the limitation of conventional analysis at the level of individual genes, analysis at the level of pathways renders great interest in providing informative insights into cellular functions that facilitates understanding of the disease mechanism. Actually, it has recently been shown that the pathway-level representation generates clinically relevant stratifications and outcome predictors for glioblastoma and colorectal cancer \cite{Drier2013Pathway} and also breast cancer \cite{Livshits2015Pathway}. Moreover, mathematical models of the activity of a pathway have demonstrated a significantly better association to poor prognosis in neuroblastoma patients than the activity of their constituent genes, including MICN, a conventional biomarker \cite{Fey2015Signaling}. This observation has recently been extended to other cancers \cite{Hidalgo2017High} and to the prediction of drug effects \cite{Amadoz2015Using}.

\begin{figure}[!htbp]
	\centering
	\includegraphics[width=0.6\textwidth]{ch5hipathia/figure/signaling_pathway}
	\caption{An illustration of cell signaling process. Typically the signal transduction begins at receptor proteins that receive molecular stimuli from cell microenvironment and ends at effector proteins that execute specific actions in response to the stimulation.}
	\label{fig5:cellsignaling}
\end{figure}

Given that the inferred activity of the pathway should be closely related to its cellular mechanism for disease progression, its use to guide cancer prognosis seems promising. Recently, a number of pathway activity inference methods have been proposed \cite{Hidalgo2017High, Jacob2012More, Li2015Subpathway, Martini2013Along}. Here, we use the canonical circuit activity analysis (CCAA) method, which has demonstrated to have a superior performance \cite{Hidalgo2017High} finding significant associations of specific circuit activities, directly responsible for triggering the prominent cancer hallmarks \cite{Hanahan2011Hallmarks}, to patient survival. This method recodes gene expression values into measurements of signaling circuit activities that ultimately account for cell responses to specific stimuli. Such activity values can be considered multigenic mechanistic biomarkers that can be used as features for cancer prognosis.

In this chapter, we demonstrate that the activity of signaling circuits yields comparable or even better prediction in breast cancer prognosis than the expression of individual genes, while detected mechanistic biomarkers enjoy the compelling advantage of readily available interpretation in terms of the corresponding cellular functions they trigger. Moreover, we show that the proposed prediction scheme can even suggest, in addition to interesting signaling circuits related to disease outcome, a list of prognostic genes that do not code for signaling proteins whose contribution to cancer prognosis potentially supplements the mechanism included in the pathways modeled. All numerical results are produced with \texttt{R} and codes for reproducing the experiments are available via \url{https://github.com/YunlongJiao/hipathiaCancerPrognosis}.




\section{Methods}
\label{sec5:methods}

\subsection{Data Source and Processing}
\label{sec5:data}

Our interest in this study lies in predicting the overall survival outcome of breast cancer patients making use of gene expression data. The breast cancer gene expression and survival data here were downloaded from The Cancer Genome Atlas (TCGA), release No. 20 of the International Cancer Genome Consortium (ICGC) data portal under project name BRCA-US\footnote{More information can be found at \url{https://dcc.icgc.org/releases/release_20/Projects/BRCA-US}.}. This dataset provides the RNA-seq counts of 18,708 genes for 879 tumor samples in which we also have records of the vital status of corresponding donors, namely the overall survival outcome of the cancer patients being alive or deceased at the end of clinical treatment (Table \ref{tab5:survival}). This way we deal with a binary classification problem distinguishing good vs poor prognosis based on gene expression measurements of breast tumor samples. Since TCGA cancer data are collected from different origins and underwent different management processes, non-biological experimental variations, commonly known as batch effect, associated to Genome Characterization Center (GCC) and plate ID must be removed from the RNA-seq data. The COMBAT method \cite{Johnson2007Adjusting} was used for this purpose. We then applied the trimmed mean of M-values normalization method (TMM) method \cite{Robinson2010scaling} for data normalization which is essential in applying the CCAA method. The resulting normalized values were finally entered to the pathway analysis method.

\begin{table}[!htbp]
	\caption{Summary of survival outcome of the breast cancer patients in the TCGA dataset.}
	\centering
	\begin{tabular}{|l|l|r|r|}
		\hline
		\bf{Donor vital status} & \bf{Pseudo label} & \bf{No. of samples} & \bf{Percentage} \\
		\hline
		Deceased (poor prognosis) & Positive & 124 & 14.1\% \\\hline
		Alive (good prognosis) & Negative & 755 & 85.9\% \\\hline
		\multicolumn{2}{|r|}{\bf{Total}} & 879 & 100.0\% \\
		\hline
	\end{tabular}
	\label{tab5:survival}
\end{table}

In order to explore the potential of utilizing external knowledge of cell signaling to enhance prognosis, we consulted Kyoto Encyclopedia of Genes and Genomes (KEGG) repository \cite{Kanehisa2012KEGG} to retrieve relationships between proteins within signaling pathways. A total of 60 KEGG pathways were used (Table \ref{tab5:kegg}), comprehending 2,212 gene products that participate in 3,379 nodes. Note that most gene products are proteins, and two types of nodes are defined in KEGG: plain nodes which may contain one or more proteins and complex nodes. These pathways each compose into a directed network where nodes are connected with edges labeled by either activation or inhibition depending on the action in transmitting signals along the path. In particular, input nodes that have no incoming edges represent receptor proteins which receive molecular stimuli from cell microenvironment, and output nodes that have no outgoing edges represent effector proteins which carry out specific cellular functions. We will elaborate in the following subsection on how to decompose the complex structure of KEGG pathways in order to effectively apply the CCAA method.


\begin{center}
	\begin{longtable}[H]{|l|l|}
	\caption{The 60 KEGG pathways for which signaling activity is modeled.}
	\label{tab5:kegg}\\
		
		\hline \multicolumn{1}{|l|}{\bf{KEGG identifier}} & \multicolumn{1}{l|}{\bf{Pathway name}} \\ \hline 
		\endfirsthead
		
		\multicolumn{2}{c}%
		{{\tablename\ \thetable{} -- continued from previous page}} \\
		\hline \multicolumn{1}{|l|}{\bf{KEGG identifier}} & \multicolumn{1}{l|}{\bf{Pathway name}} \\ \hline 
		\endhead
		
		\multicolumn{2}{|r|}{\textit{Continued on next page}} \\ \hline
		\endfoot
		
		\hline
		\endlastfoot
		
		hsa04014 &
		Ras signaling pathway \\\hline
		hsa04015 &
		Rap1 signaling pathway \\\hline
		hsa04010 &
		MAPK signaling pathway \\\hline
		hsa04012 &
		ErbB signaling pathway \\\hline
		hsa04310 &
		Wnt signaling pathway \\\hline
		hsa04330 &
		Notch signaling pathway \\\hline
		hsa04340 &
		Hedgehog signaling pathway \\\hline
		hsa04350 &
		TGF-beta signaling pathway \\\hline
		hsa04390 &
		Hippo signaling pathway \\\hline
		hsa04370 &
		VEGF signaling pathway \\\hline
		hsa04630 &
		Jak-STAT signaling pathway \\\hline
		hsa04064 &
		NF-kappa B signaling pathway \\\hline
		hsa04668 &
		TNF signaling pathway \\\hline
		hsa04066 &
		HIF-1 signaling pathway \\\hline
		hsa04068 &
		FoxO signaling pathway \\\hline
		hsa04020 &
		Calcium signaling pathway \\\hline
		hsa04071 &
		Sphingolipid signaling pathway \\\hline
		hsa04024 &
		cAMP signaling pathway \\\hline
		hsa04022 &
		cGMP-PKG signaling pathway \\\hline
		hsa04151 &
		PI3K-Akt signaling pathway \\\hline
		hsa04152 &
		AMPK signaling pathway \\\hline
		hsa04150 &
		mTOR signaling pathway \\\hline
		hsa04110 &
		Cell cycle \\\hline
		hsa04114 &
		Oocyte meiosis \\\hline
		hsa04210 &
		Apoptosis \\\hline
		hsa04115 &
		p53 signaling pathway \\\hline
		hsa04510 &
		Focal adhesion \\\hline
		hsa04520 &
		Adherens junction \\\hline
		hsa04530 &
		Tight junction \\\hline
		hsa04540 &
		Gap junction \\\hline
		hsa04611 &
		Platelet activation \\\hline
		hsa04620 &
		Toll-like receptor signaling pathway \\\hline
		hsa04621 &
		NOD-like receptor signaling pathway \\\hline
		hsa04622 &
		RIG-I-like receptor signaling pathway \\\hline
		hsa04650 &
		Natural killer cell mediated cytotoxicity \\\hline
		hsa04660 &
		T cell receptor signaling pathway \\\hline
		hsa04662 &
		B cell receptor signaling pathway \\\hline
		hsa04664 &
		Fc epsilon RI signaling pathway \\\hline
		hsa04666 &
		Fc gamma R-mediated phagocytosis \\\hline
		hsa04670 &
		Leukocyte transendothelial migration \\\hline
		hsa04062 &
		Chemokine signaling pathway \\\hline
		hsa04910 &
		Insulin signaling pathway \\\hline
		hsa04922 &
		Glucagon signaling pathway \\\hline
		hsa04920 &
		Adipocytokine signaling pathway \\\hline
		hsa03320 &
		PPAR signaling pathway \\\hline
		hsa04912 &
		GnRH signaling pathway \\\hline
		hsa04915 &
		Estrogen signaling pathway \\\hline
		hsa04914 &
		Progesterone-mediated oocyte maturation \\\hline
		hsa04921 &
		Oxytocin signaling pathway \\\hline
		hsa04919 &
		Thyroid hormone signaling pathway \\\hline
		hsa04916 &
		Melanogenesis \\\hline
		hsa04261 &
		Adrenergic signaling in cardiomyocytes \\\hline
		hsa04270 &
		Vascular smooth muscle contraction \\\hline
		hsa04722 &
		Neurotrophin signaling pathway \\\hline
		hsa05200 &
		Pathways in cancer \\\hline
		hsa05231 &
		Choline metabolism in cancer \\\hline
		hsa05202 &
		Transcriptional misregulation in cancer \\\hline
		hsa05205 &
		Proteoglycans in cancer \\\hline
		hsa04971 &
		Gastric acid secretion \\\hline
		hsa05160 &
		Hepatitis C
	\end{longtable}
\end{center}

\subsection{Modeling Framework for Signaling Pathways}
\label{sec5:ccaa}

We applied the canonical circuit activity analysis (CCAA) method\footnote{Implementation available at \url{https://github.com/babelomics/hipathia}.} proposed by \cite{Hidalgo2017High} in pursuit of modeling signaling activity. Overall, CCAA is a method that estimates the level of activity within a signaling circuit by modeling cell signaling process in order to recode gene expression values into measurements that ultimately account for cell responses caused by pathway activities. Essentially the CCAA method computes an activity value for each stimulus-response sub-pathway within signaling circuits. This way, the sub-pathways which associate naturally with cell functionalities can be considered as mechanistic features that are modularized from multigenic signatures, and their activity values connected to the activation or deactivation of specific cellular functions thus provide quantitative clues to understand disease mechanisms when further related to phenotypes of interest such as cancer survival.

Recall that in cell signaling process represented in KEGG pathways, cell signal arrives to an initial input node and starts to transmit along any path following the direction of the edges until it reaches an output node that finally triggers a cellular action. In particular, from different input nodes the signal may follow different routes to reach different output nodes. Within the modeling context, a \textit{canonical circuit} is naturally defined as all possible routes the signal can traverse to be transmitted from a particular input node to a particular output node (Figure \ref{fig5:pathway}, A). A total of 6,101 canonical circuits are identified and modeled in this study. Now we take efforts to describe first how CCAA estimates the signaling activity of a canonical circuit.

\begin{figure}[!htpb]
\centering
\includegraphics[width=\textwidth]{ch5hipathia/figure/pathway}
\caption{The different levels of abstraction within pathways: A) Canonical circuits that communicate one receptor to one effector; B) Effector circuits that communicate all the receptors that signal a specific effector; C) Function circuits that collect the signal from all the effectors that trigger a specific function (according to UniProt or GO keywords); D) Cancer hallmarks, a sub-selection of only those functions related to cancer hallmarks.}
\label{fig5:pathway}
\end{figure}

In a signaling circuit, the transmission of the signal depends on the integrity of the chain of nodes that connect the receptor to the effector and the capability of transmitting signals of each node involved intuitively depends on two folds: the abundance of the proteins corresponding to that node and its activity status due to the interaction with its parent nodes. First, we need to estimate a value for each node in the pathways in regard to the presence of proteins involved in the node. Following the convention of \cite{Bhardwaj2005Correlation, Efroni2007Identification, Montaner2009Gene, Sebastian-Leon2014Understanding}, the presence of the mRNA (the normalized RNA-seq counts rescaled between 0 and 1) is taken as a proxy for the presence of the proteins involved in each node. Notably, for different types of nodes defined in KEGG, the value of a plain node in the pathways is defined as the ninetieth percentile of the values of the proteins contained, and the value of a complex node is taken as the minimum value of the proteins contained (the limiting component of the complex). Then, the degree of integrity of the circuit is estimated by modeling the signal flow across it, transmitting node-by-node following the path while its intensity value gets propagated along the way taking into account the current node value and the intensity of the signals arriving to it. Specifically, we initialize an incoming signal of intensity value of $1$ received by the input (receptor) node of the circuit, and then for each node $n$ of the circuit, the signal value $s_n$ is updated by the following rule:
$$
s_n = v_n \cdot \left( 1 - \prod_{a \in A_n} (1-s_a) \right) \cdot \prod_{i \in I_n} (1-s_i) \,,
$$
where $A_n$ denotes the set of signals arriving to the current node $n$ from activation edges, $I_n$ denotes the set of signals arriving to the current node $n$ from inhibition edges, and $v_n$ is the (normalized) value of the current node $n$. In case of loops present in the circuit, a node may be visited multiple times, until the difference in the updates of the signal value at that node is below certain threshold, before the signal exits the loop and continues to propagate down the cascade. Finally, the activity value for the circuit is defined by the signal intensity transmitted through the last (effector) protein of the circuit which quantifies the cell function ultimately activated by the circuit. See Figure \ref{fig5:ccaa} for an example of deducing the activity value of an artificial canonical circuit by the CCAA method.

\begin{figure}[!htpb]
	\centering
	\includegraphics[width=0.6\textwidth]{ch5hipathia/figure/pretty_algorithm2}
	\caption{An example of computing the activity value of an artificial canonical circuit by the CCAA method. In Step 1, node values are derived from the normalized mRNA measurements. In Step 2, signal is propagated along the path while its intensity value gets updated according to the rule of the CCAA method. Finally, The signal value attained after the last protein is visited accounts for the signaling activity of the circuit.}
	\label{fig5:ccaa}
\end{figure}

Besides, the CCAA method straightforwardly allows to explore pathway-level analysis at different levels of abstraction by applying to different notions of signaling circuits. As the output nodes at the end of canonical circuits are the ultimate responsible to carry out specific cellular actions, an \textit{effector circuit} is defined from a functional viewpoint as a higher-level signaling entity that compose all canonical circuits ending at the same output node (Figure \ref{fig5:pathway}, B). When applied to an effector circuit, the CCAA method returns the joint intensity of the signal arriving to the corresponding effector node. Furthermore, the known functions triggered in cell by each effector protein can be derived from their functional annotations. Here we use UniProt \cite{Consortium2015UniProt} and Gene Ontology (GO) \cite{Consortium2015Gene} annotations. Finally, inferred signaling activity values of those effector circuits ending at proteins with the same annotated functions are averaged to quantify the activity of the function realized in cell. This way we obtain estimated activity values directly connected to a list of cellular functions (Figure \ref{fig5:pathway}, C). Figure \ref{fig5:pathway} depicts the different levels of abstraction from canonical circuits, to effector circuits and finally functions. Eventually for the sake of interpretation, a subset of curated functions can be used for a specific scenario in which the relevant functions are known to interpret the cancer biology, for which we use cancer hallmarks \cite{Hanahan2011Hallmarks} (Figure \ref{fig5:pathway}, D).

\subsection{Cancer Prognosis with Inferred Signaling Pathway Activity}
\label{sec5:prognosis}

In this study, we are interested in comparing the prognostic power of pathway-level mechanistic features and gene-based features, both separately and in combination, in order to distinguish good vs poor prognosis. Using the CCAA method, we recoded the list of gene expression values of each tumor sample into the corresponding lists of signaling activity values for the three levels of abstraction: canonical circuits, effector circuits and functions, as described in UniProt and GO annotations. Therefore for each tumor sample, we end up with a profile of gene expression, a profile of canonical circuit signaling activity, a profile of effector circuit signaling activity, a profile of UniProt-based cellular function activity and a profile of GO-based cellular function activity. These profiles are sample-specific, or so-called \textit{personalized}, profiles that can be straightforwardly used as prognostic features for cancer prognosis following any off-the-shelf classification algorithm. Note that pathway-level profiles are derived with no regard to any information provided by the genes whose products do not participate in cell signaling, and the prognostic power of pathway-level profiles may thus be limited by the coverage of genes in known biological pathways. In order to understand the relative contribution to the pathway-level profiles and gene-level profiles to the accurate separation between good vs poor prognosis, we devised 4 artificial profiles: path-gene expression profile containing only genes that are involved in the KEGG signaling pathways, other-gene expression profile containing only genes that are absent from the KEGG pathways, a combined profile consisting of signaling activity of effector circuits and expression of other-genes, and a combined profile consisting of signaling activity of canonical circuits and expression of other-genes. Thus we obtained a total of 9 types of profiles (detailed in Table \ref{tab5:profile}).

\begin{table}[!htpb]
	\caption{Summary of different types of profiles used as predictive features for breast cancer prognosis.}
	\centering
	\begin{tabular}{|p{2cm}|p{3.5cm}|>{\raggedleft\arraybackslash}p{1.5cm}|p{4cm}|}
		\hline
		\bf{Alias} & \bf{Profile type} & \bf{No. of features} & \bf{Analysis level} \\\hline
		fun.vals & UniProt-based functions & 81 & \multirow{2}{*}{\parbox{4.5cm}{\raggedright Pathway-level cellular function values}} \\
		go.vals & GO-based functions & 370 &  \\
		\hline
		eff.vals & Effector circuits & 1,038 & \multirow{2}{*}{\parbox{4.5cm}{\raggedright Pathway-level signaling activity values}} \\
		path.vals & Canonical circuits & 6,101 &  \\
		\hline
		path.genes.vals & Path-genes & 2,212 & \multirow{3}{*}{\parbox{4.5cm}{\raggedright Gene-level expression values}} \\
		other.genes.vals & Other-genes & 16,496 &  \\
		genes.vals & All genes & 18,708 &  \\
		\hline
		eff.and.other.-genes.vals & Effector circuits and other-genes & 17,534 & \multirow{2}{*}{\parbox{5cm}{\raggedright Combination of pathway-level signaling activity values and gene-level expression values}} \\
		path.and.other.-genes.vals & Canonical circuits and other-genes & 22,597 &  \\
		\hline
	\end{tabular}
	\label{tab5:profile}
\end{table}

From the viewpoint of machine learning, this study is formulated as a typical binary classification problem where we determine a positive or negative pseudo label for each sample. Based on the data available in this study (Table \ref{tab5:survival}), we perform a 5-fold cross-validation repeated 10 times on the dataset and report the mean performance over the $5 \times 10 = 50$ splits to assess the prognostic power for each type of profile. The performance is evaluated by the Area Under the ROC Curve (AUROC) criteria \cite{Sing2005ROCR}. Note that usually a classifier returns a continuous prediction between 0 and 1 for each sample denoting the probability of that sample being in the positive class rather than in the negative class, and then assigns either label to the sample according to some cutoff value thresholding the prediction. AUROC is a cutoff-free score that measures the probability that the classifier will score a randomly drawn positive sample higher than a randomly drawn negative sample.

In this study, we considered a total of 12 classification algorithms as candidate classifiers, most of which are state-of-the-art (Table \ref{tab5:classifier}). When we assess the prognosis performance for a specific type of profile on a specific (external) cross-validation split of the data, we perform an internal 5-fold cross-validation on the training set to determine which classifier returns the highest cross-validated performance and the best classifier is then used on the test set to obtain the performance score. The rationale behind the nested cross-validation is that, although any classification algorithm from the machine learning literature can be used to discriminate good vs poor prognosis with any profile type considered as predictive features, in practice, however, we do not have a definitive concept of which classifier suits the best universally for all types of profiles. In other words, it will be a confusing factor if we predetermine just one classifier throughout the study. In fact, the underlying hypotheses of different classifiers vary, for instance linear or non-linear relationships can be assumed between features and labels, and some classifiers can be particularly sensitive to the presence of a large number of noisy features. As a consequence, the procedure of choosing the best suited algorithm for different types of profiles by a nested cross-validation guarantees that the prediction performance is evaluated in an impartial fashion.

\begin{table}[!htpb]
	\caption{The 12 candidate classifiers used to discriminate prognosis classes for breast tumor samples.}
	\centering
	\begin{tabularx}{\textwidth}{|l|X|X|}
		\hline
		\bf{Alias} & \bf{Classifier} & \bf{Reference} \\\hline
		LDA & Linear discriminant analysis & \cite{Venables2002Modern, Ripley2007Pattern} \\\hline
		LogitLasso & L1-regularized logistic regression & \cite{Friedman2010Regularization} \\\hline
		LinearSVM & Support Vector Machine with linear kernel & \cite{Chang2011LIBSVM} \\\hline
		RadialSVM & Support Vector Machine with Gaussian RBF kernel & \cite{Chang2011LIBSVM} \\\hline
		KendallSVM & Support Vector Machine with Kendall kernel & \cite{Zeileis2004kernlab, Jiao2015Kendall} \\\hline
		KNN & $k$-nearest neighbor classifier & \cite{Venables2002Modern, Ripley2007Pattern} \\\hline
		NB & Naive Bayes classifier & \cite{Ripley2007Pattern} \\\hline
		GBM & Gradient Boosting Machine & \cite{Friedman2001Greedy} \\\hline
		RF & Random Forest & \cite{Liaw2002Classification, Breiman2001Random} \\\hline
		SparseSVM & L1-regularized L2-loss Support Vector Machine & \cite{Fan2008LIBLINEAR} \\\hline
		PAM & Nearest shrunken centroid classifier & \cite{Tibshirani2002Diagnosis} \\\hline
		Constant & Majority voting classifier & Outputs constant label of the dominant class (negative-control) \\\hline
	\end{tabularx}
	\label{tab5:classifier}
\end{table}

