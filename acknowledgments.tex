First and foremost, I would like to thank Jean-Philippe Vert for being an inspiring advisor and a supportive supervisor, for having welcomed me into CBIO with an amazing funding opportunity, for sharing his experience and ideas with me that have both consciously and unconsciously shaped my academic and communication skills, for setting an excellent example of a researcher with contagious enthusiasm and a group leader with motivational leadership, without whom I would simply never have completed this thesis.


During my PhD, I was fortunately offered the opportunity to work with Joaquin Dopazo, who proposed and led one of my doctoral projects and mentored me with encouragement and inspiration during my stay at CIPF, and Stefan Kobel, who patiently trained my presentation skills and enriched my background knowledge in biochemistry during my stay at Roche; both mentors have been huge influences to me, for which I cannot express enough gratitude.


Many other people have contributed, directly or indirectly, to the work presented in this thesis, and I would like to thank: Elsa Bernard, Erwan Scornet, V\'{e}ronique Stoven and Thomas Walter, for participating in the DREAM challenge as a team; Fabian Heinemann and Sven Dahlmanns, for the discussion on the project of analyzer failure prediction; Eric Sibony and Anna Korba, for suggesting and collaborating on the project of rank aggregation; Marta Hidalgo, Cankut \c{C}ubuk, Alicia Amadoz, Jos\'{e} Carbonell-Caballero and Rub\'{e}n S\'{a}nchez, for the comments and help on the project of network analysis; last but not least, Vincent Brunet, for always being so responsive and helpful whenever I had an embarrassingly trivial problem with the server.


I would also like to thank my thesis reviewers, Risi Kondor and St\'{e}phan Cl\'{e}men\c{c}on, for their time, interest and helpful comments, and other members on my defense jury, Chlo\'{e}-Agathe Azencott, Francis Bach, Joaquin Dopazo and Jean-Philippe Vert, for their time and insightful questions.


The few people who have been an immensely significant part of my professional and personal life during my PhD must be specially mentioned, in that my unexpected encounter with them and their involvement in my life afterwards can only be described by no better words than \textit{kizuna} (a special bond of friendship).


MeiMei channn, thank you for being the first person who ever talked to me at our first ITN summer school in T\"{u}bingen and then becoming one of my closest friends two years later at the fourth time we met, for those countless times of selflessly helping and teaching me with programming, biology and everything you know, for always being there for me caring every little thing happening around me, for listening to my joys and misery and also sharing happiness and frustration in life, for having never complained about my constant complaints and never been bored of my tedious stories, for every moment during the very few times we could meet that I cherish for the rest of my life, or simply for agreeing instantaneously to have more than two dinners until we got totally bloated every time we hang out.


Puppy Peeter, thank you for showing up in CBIO since when the lab just seemed to me a much different place to be in, for helping Google translate all the abstracts in this thesis into French together with lovely Lucile, for having the best taste in food, except for cheese, and sharing as much interest in burgers as I do, for being the first one and the only one in the lab for a long time who would think calling me by a different name was not an inappropriate thing, or simply for being so adorable to talk to, to be around with or just to look at.


Cankut, thank you for being such a great labmate, flatmate and frrriend during my stay in Valencia, for showing me around so many times that had made me fall in love with every bit of the city, for being one of the most truly selfless and genuinely sympathetic people I know, for having the cutest beagle in the world, MoMo, who would lick me every morning until I woke up, or simply for being one of the most important reasons that my stay in Spain was such an unforgettable experience that I keep going around telling everyone how much it means to me.


MI LOBE SE\~{N}ORITO, shank you for being Shpanish first of all, then for teaching me sho much matsch, including eigenvaluesh in particular, and influenshing me with your shrewd wishdom in life, for trushting me blindfolded and opening up to me sho easily that makesh me feel sho very shpecial, for making me shunny in a gloomy placshe without even having to try, for putting up with my shilliness and grumpiness and even being shilly and grumpy together, or shimply for running down with me to my favorite reshtaurant in Parish every Tueshday, but the one shing I am not at all shankful for is how late you came in my life when I will have to leave shoon.


jacoPoo, thank you for having made my last year in Paris so wOndErfUl, for bringing up the Italian soul in me by giving me an Italian name now everyone knows me by and teaching me how to speak with a hand, for showing me the aesthetic side of you that enlightens my capacity for art, for always understanding me and sticking with me under any circumstances, or simply for being my brother from another mother who made me leaving Paris so much harder than it should have been.


Ana! Thank you too for having made my last year in Paris so special, for having the unique personality of being the meanest on the outside and sweetest on the inside, for bringing the competition of being shameless to another level for me, for not only enduring but treasuring the superficiality and stupidity of me, or simply for being my sister from another mother who too made me leaving Paris so much harder than it should have been.


I would also like to thank all the former and current members of CBIO: Nelle, for having helped me a lot during and especially at the beginning of my PhD as an admirable \textit{senpai} (a senior colleague) to me, for giving me plenty of valuable advice on building a professional career, and for co-founding the CBOG (CBIO Beer Organizing Group); V\'{e}ronique, for being one of the most optimistic and delightful people I know who tells the funniest stories non-stop while being a respectable professor; Thomas, for being a determined researcher and a motivating character to me; Chlo\'{e}, for giving a lot of helpful comments and advice on learning with networks, and for setting up an outstanding example for me as a researcher with a successful career established at my age; Victor, for being the only other person in the lab who did not speak French for three years; Nino, for being so kind and encouraging all the time with whom I could talk about science, even comfortably in French; Marine, for having to sit in the same office with me with whom I could ``professionally'' and casually talk to from time to time at work; Beyrem, for being a funny guy; Xiwei, for so many pieces of important information that I managed to not have myself evicted by the French prefectures; Beno\^{i}t, for accomplishing the mission impossible that you had single-handedly changed what Paris and France meant to me; Joe, for being such an adorable human being I like to hang out a lot with but at the same time such an annoying yet weirdly charming one I can never really get mad at; and many others from CBIO I will apologetically skip naming, for the enjoyable moments and pleasant conversations over a cup of coffee or a pint of beer occasionally. Besides the regulars, Ramona, Ilaria and several other visitors brought appreciable dynamic to CBIO, for which I am very grateful.


During my secondments in Germany and in Spain, many people came across that made my short stay abroad much less lonely, and I would deeply thank: Kathrin, Miaolin and others colleagues from Roche, for their amiable company in Penzberg; Edgar, Carol, Pau, Kinza, Sema, Julen and other colleagues from CIPF, for their delightful friendship in Valencia, especially outside of the lab, and Javi, for hosting me in his apartment with enormous generosity and warm-heartedness when I went to Madrid for visa affairs.


Finally, this thesis is dedicated to the most important people in my life even though they would never have read this, my parents, for always believing in me since I was born, for having supported every single decision I made, for raising me up and providing everything I needed but never asking anything in return.


Funding-wise, my PhD was supported by the European Union 7th Framework Program through the Marie Curie Initial Training Network (ITN) Machine Learning for Personalized Medicine (MLPM) grant No. 316861, and by the European Research Council grant ERC-SMAC-280032.


\begin{flushright}
Yunlong
\\
July 2017, in Paris
\end{flushright}
