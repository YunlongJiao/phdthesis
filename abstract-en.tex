Breast cancer is the second most common cancer worldwide and the leading cause of women's death from cancer. Improving cancer prognosis has been one of the problems of primary interest towards better clinical management and treatment decision making for cancer patients. With the rapid advancement of genomic profiling technologies in the past decades, easy availability of a substantial amount of genomic data for medical research has been motivating the currently popular trend of using computational tools, especially machine learning in the era of data science, to discover molecular biomarkers regarding prognosis improvement. This thesis is conceived following two lines of approaches intended to address two major challenges arising in genomic data analysis for breast cancer prognosis from a methodological standpoint of machine learning: rank-based approaches for improved molecular prognosis and network-guided approaches for enhanced biomarker discovery. Furthermore, the methodologies developed and investigated in this thesis, pertaining respectively to learning with rank data and learning on graphs, have a significant contribution to several branches of machine learning, concerning applications across but not limited to cancer biology and social choice theory.