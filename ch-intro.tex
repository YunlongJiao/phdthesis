\chapter{Introduction}
\label{chap:intro}

Breast cancer is one of the most common malignancy for women and breast cancer prognosis is currently one of the leading problems of great interest. The advent of high-throughput platforms for gene expression profiling motivates the popular trend of using molecular biomarkers for clinical decision making regarding prognosis improvement assisting therapeutic strategies. It has been widely recognized as a very promising yet challenging problem to extract potentially valuable information from genomic data. The relatively small number of clinical samples, the inherent measurement noise in high-throughput experimental data and the heterogeneity across patients make it demanding to detect reliable biomarkers.

MammaPrint \cite{VantVeer2008Enabling}.

An important feature of the proposed scheme is the idea of promoting gene-level analysis to pathway-level analysis by applying the \textit{hiPathia} method to obtain patient-specific personalized profiles of signaling circuit activity. Over the past decade, a great deal of efforts have been devoted to analyzing gene expression data in extensively diverse biomedical applications revolving around cancer as well as other diseases. Typical examples include cancer prognosis and subtyping, patient stratification and biomarker detection. Depending on the specific problem at hand, many existing algorithms that were originally designed for gene expression profiles can be easily adapted to signaling circuit activity profiles. Consequently the analysis is made at the level of functional pathways instead of individual genes, often leading to better understanding in terms of biological relevance.


%\section{Illustration Example}
%
%\subsection{A subsection just for fun}
%
%Sorry I won't write your PhD here ;) This small text just to mention that this style supports writing with accents such as in french words (th�se, d�finir, ...). Also I put here a simple way to include an image. This is standard latex. For pdflatex compilation, the extension of the images is jpg. For latex compilation, this is ps or eps. The base folder containing images is set in formatAndDefs.tex, as well as the default extensions added to the image names.
%
%\begin{figure}[!htbp]
%  \begin{center}
%    \includegraphics[width=0.9\textwidth]{ch-intro/arctic_control}
%  \end{center}
%  \caption{A nice image...}
%  \label{fig:jolieImage}
%\end{figure}
%
%\section{An equation}
%
%Just to show argmin and partial derivative commands.
%
%\section{An other section}
%
%Showing a great bullet list environment:
%
%\begin{bulletList}
% \item First point
% \item Second point
%% \item Here is an abbreviation reference \nomenclature{DTI}{Diffusion Tensor Imaging} DTI
%\end{bulletList}
