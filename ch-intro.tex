\chapter{Introduction}
\label{chap:intro}

\begin{chapabstract}

\textrm{{\bf Abstract:}} This thesis work has been driven typically by the development and investigation of machine learning methods in genomic data analysis for improved breast cancer survival prediction and for detecting robust molecular signatures related to cancer prognosis. Towards these goals, two directions in terms of methodologies have been explored for supervised survival prediction and structured feature selection, which are rank-based and network-guided approaches. This chapter serves as a general introduction to the thesis that briefly summarizes the background of breast cancer prognosis and summarizes the motivation of my thesis research to address the computational challenges in genomic data analysis for cancer prognosis.
\linebreak
\vskip 0.1in
\noindent \textrm{{\bf R�sum�:}}

\end{chapabstract}

\section{General Background}



% copied : Breast cancer is the most commonly diagnosed cancer in women in Europe and it is the first cause of women�s cancer death in Europe. In Finland in 2011 about 4900 new breast cancers were found. The risk of breast cancer in women has increased continuously and the rate of cancer patients is growing further. The incidence of breast cancer is very low among women under 30 years, but the risk of developing cancer increases after 45 years of age. One in nine women will have breast cancer at some point in her life and one in thirty will die from the disease. Even if the number of new breast cancer cases has multiplied during the past decades, breast cancer survival has improved over time possibly due to early diagnosis by effective screening programs and advances in treatment.

breast cancer is malignant and complex... many deaths... disease severity example and statistics... % copied : Cancer is one of the leading causes of death worldwide. The main types of cancer are lung, prostate, colorectal and breast cancer (WHO, 2011), the last one being the most common cause of cancer mortality among women worldwide (Chodosh, 2011).


The initial treatment for breast cancer patient is usually accomplished by complete removal of tumor by surgery or radiation without damage to the rest of the body. After the initial treatment, many patients are advised to receive systemic treatment including adjuvant chemotherapy, hormone therapy and targeted therapy to lower the risk of relapse. The decision is made based on prognosis, estimation of the risk of relapse or likely course of outcome after initial treatment if no adjuvant treatment is given. it is important to identify those patients who are likely to benefit the most from treatment.\footnote{Difference between prognosis and prediction (responsiveness and effect of certain treatment); usually discussed together but omitted from the present thesis for ease of emphasis.} For example, as the most common type, adjuvant chemotherapy are usually cytotoxic drugs with strong deleterious side effect... as aggressive treatments are most beneficial for high risk of relapse and minimizing the unnecessary intake of adjuvant chemotherapy is of chief importance... therefore accurate prognosis for each individual patient is important so as to personalize the best therapeutic option...




due to complexity and diversity of cancer, it is challenging to accurately do prognosis or identify prognostic factors for each particular case... conventionally prognosis of breast cancer is based solely on clinico-pathological features... Several well-established prognostic factors are commonly used to guide the clinical management of breast cancer (ref table)... a few words of subtypes and treatment/survival... well-known 6 prognostic factors used by Adjuvant! Online\footnote{www.adjuvantonline.com} : patient age, tumor size, grade, hormone receptor status, number of positive lymph nodes and comorbidity level... 


Table: Clinico-pathological features commonly used as prognostic factors of breast cancer. (saini's thesis table 1.1, pia's thesis table 1)...


however, patients of similar clinico-pathological type can still have remarkably different prognosis... according to a study in \cite{Veer2008Enabling}, in the US approximately up to 70\% of breast cancer patients with negative lymph nodes and a tumour size 1 to 2 cm undergo a toxic therapy that they would very likely not benefit from after they were actually already cured of their disease... indicates that the clinico-pathological information alone is far from sufficient to characterize the outcome of each diseased individual. Difficulties in detecting reliable and efficient prognostic factors for breast cancer remains in the heterogeneity across patients.






\section{Genomic Data Analysis for Cancer Prognosis: Prospects and Challenges}




It has been long established that genomic features contain unique characteristics of each patient and offer the opportunity of scrutinizing the individuality of each breast tumor. Often termed by \textit{biomarker} are such molecular-level prognostic factors that characterize intrinsic biological heterogeneity of tumours.  The characterization of such biomarkers for breast cancer prognosis include gene expression, SNPs, etc... many have been associated with breast cancer survival (ref table)...


Table: Some important genes whose mutation carriers are associated with high risk for breast cancer. (Pia's thesis table 2)


The advancement of genomic profiling technologies in the past decades has been motivating the trend of using genomic information in decision-making regarding prognosis improvement, as millions of genomic features can be efficiently collected from patients and tumors. Take as an example, high-throughput gene expression profiling technologies: DNA microarray and RNA-seq, NGS, WGS...



Genomic data analysis is a broad topic... In the genre of genomic data analysis for cancer prognosis, two major questions need to be addressed: 


1) survival risk prediction via supervised statistical inference...


prognosis in the language of machine learning - classification or survival analysis... four types of survival : DMFS, RFS, DFS, OS (saini's thesis table 1.1); binary labels converted from survival... feature selection (2nd aspect below) or not (whole genome)...



2) biomarker discovery via feature selection...

Promising results with a few genomic tests available... Mammaprint \cite{Veer2008Enabling} and other tests http://www.breastcancer.org/symptoms/testing/types/mammaprint






Despite the successful stories, it has been widely recognized as a very challenging problem to extract potentially valuable information from genomic data. Besides the complexity and heterogeneity of the disease, relatively small number of clinical samples, the inherent measurement noise in high-throughput experimental data, the heterogeneity of patient data makes it demanding... furthermore, robustness of the identified biomarkers and \textit{a posteriori} interpretation of the findings are also major concerns... literature review???






\section{Contribution of the Thesis}


\subsection{Rank-based Approaches for Improved Survival Risk Prediction}


This thesis is conceived following the two lines of ideas of performing genomic data analysis for cancer prognosis. The first line of ideas is to perform gene expression data analysis based solely on the ranking of the expression levels of multiple genes while their real-valued measurements are disregarded. It is based on the assumption that the relative ordering of gene expression levels can be potentially more informative than their real-valued measurements when used in some biomedical applications since high-throughput high-dimensional genomic data are often subject to high measurement noise. From the point view of machine learning, it reduces to the study of a particular type of structured data, specifically rankings. It is well-known that kernel methods have found many successful applications where the input data are discrete or structured including strings and graphs. The first project of my doctoral studies was focused on proposing computationally attractive kernels for rank data and applying kernel methods to problems involving rankings. Central to this work was the observation that the widely used Kendall tau correlation and the Mallows similarity measures are indeed computationally attractive positive definite kernels for rankings. These kernels were further tailored for more complex types of rank data that prevail in real-world applications, especially when rankings come from real-valued vectors by keeping simply the relative ordering of the values of multiple features thereof. Thanks to these kernels, many off-the-shelf kernel machines are available to solve various problems at hand. It is worth special mention that, despite that the project was initially motivated by biomedical applications, the prospective contribution of this work concerns applications from many fields of machine learning pertaining to learning from rankings, or learning to rank. This study will be presented in Chapter \ref{chap:kendall}.



The study of the Kendall kernel for rankings has paved an unprecedented way towards a deeper understanding of a classical problem called Kemeny aggregation from the field of Social Choice Theory. Kemeny aggregation searches for a consensus ranking that best represents a collection of individual rankings in the sense that the sum of the Kendall tau distance between each ranking and the consensus is minimized. Kemeny aggregation is often considered to provide the ideal solution among all ranking aggregation criteria but the Kemeny consensus is known to be NP-hard to find. Many tractable approximations to the Kemeny consensus have therefore been proposed and well studied. Since the Kendall kernel derives from an inner product of a Euclidean space, the Kendall tau distance derives from a squared Euclidean distance and thus the combinatorial problem of Kemeny aggregation is endowed with a novel intuitive interpretation from a geometric point of view. Based on this observation, a tractable upper bound of the estimation error in terms of the distance between the Kemeny consensus and its approximation is established. This upper bound requires little assumption on the approximation procedure or the collection of rankings to aggregate. However, due to its remote connection to cancer prognosis that is the primary objective of the present thesis, this study will be presented in Appendix \ref{chap:kemeny}. 



\subsection{Network-guided Approaches for Enhanced Biomarker Discovery}


The second line of ideas of performing genomic data analysis for cancer prognosis is to consult biological networks as domain-specific knowledge to guide our analysis. In fact, biological networks are a common way of depicting functional relationships between genes that have been accumulated from decades of biomedical research, and they can be particularly valuable when incorporated as prior knowledge in order to improve the selection efficacy of prognostic biomarkers related to breast cancer prognosis. Typical examples of biological networks include protein-protein interaction network and signaling pathway network.


Two projects were initiated on network-guided gene expression data analysis for breast cancer prognosis and for identification of genomic signatures related to cancer survival. In one project, we focused on the study of structured regularization in generalized linear models for selecting prognostic genes where the regularization method is designed so that genes closer on the biological network are encouraged to be selected simultaneously. In fact, in order to achieve such simultaneous modularity and sparsity coherent with respect to presumed network structure, a popular method called network-based wavelet smoothing has been successfully applied in many applications from the field of signal processing. Therefore, we were motivated to investigate the potential of this method in biomedical applications, using a breast cancer gene expression dataset and a protein-protein interaction network to guide our analysis albeit the methodology is generally applicable to various types of genomic data and biological networks. In particular, the method allows to designate a few prognostic gene modules with intra-collaborative functionality rendering readily interpretable insights potentially related to cancer survival. Numerical results demonstrated that, compared to several network-free and some established network-based regularization methods, network-based wavelet smoothing was able to improve the selection efficacy of prognostic genes in terms of stability, connectivity and interpretability, while achieving competitive performance of survival risk prediction for breast cancer. This study will be presented in Chapter \ref{chap:wavelet}.


In another project, we focused on a particular type of biological network namely signaling pathway network. Based on a modeling framework of cell signaling, gene expression profiles can be translated into personalized profiles of signaling pathway activities by integrating signaling pathway network. When gene-level profiles are replaced by these derived pathway-level profiles as input to many off-the-shelf computational tools, a simple scheme emerges where gene-level analysis is easily promoted to pathway-level analysis. Analysis made at the level of functional pathways consequently leads to better understanding in terms of biological relevance. Notably, when combined with feature selection oriented algorithms, the proposed scheme enables direct identification of pathway-level mechanistic signatures as an alternative to conventional gene-based signatures, which provides more informative insights into the cellular functions involved in cancer survival. This study will be presented in Chapter \ref{chap:hipathia}.




\subsection{Other Contribution}


During the course of my doctoral studies, I have undertaken some other projects as well. In 2013, Elsa Bernard, Erwan Scornet, V\'{e}ronique Stoven, Thomas Walter, Jean-Philippe Vert and I from our laboratory participated as a team in an international bioinformatics competition called \textit{DREAM 8 NIEHS--NCATS--UNC Toxicogenetics Challenge}. In the competition, participants were asked to predict the response of human cell lines exposed to various toxic chemical compounds, based on the chemicals' molecular characterization and the cell lines' transcriptome. Finally our team won second place with a kernel bilinear regression model. Oral presentation was accepted to \textit{NIPS Workshop on Machine Learning in Computational Biology (MLCB)} and later invited to \textit{RECOMB Conference on Regulatory and Systems Genomics}. This work has been published as part of the crowd-source collaboration as a result of the competition \cite{Eduati2015Prediction} but excluded from the present thesis.



During the spring of 2015 when I was interning at Roche Diagnostics, Penzberg, Germany, I worked on failure state prediction for automated analyzers for analyzing biological samples, and a European patent application regarding the methodology was filed by Roche in December 2016 and is currently pending approval \cite{Jiao2016Failure}. Due to corporate confidentiality issues, this study will not be included in the present thesis.

